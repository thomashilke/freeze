\section{Introduction}
Le package \texttt{freeze} impl\'emente une m\'ethode num\'erique de
type \'el\'ements finis et une formule de Chernoff pour approximer la
solution d'un probl\`eme de Stefan en une dimension d'espace.

\section{Formulation math\'ematique du probl\`eme}
Soit $\rho$ la densit\'e suppos\'ee constante d'un mat\'eriau qui
subit une transition entre le phases solide et liquide \`a la
temp\'erature $\Theta_t$. La chaleur sp\'ecifique du mat\'eriau dans
les phases solide et liquide est not\'ee respectivement $C_{p,1}$ et
$C_{p,2}$. Soit $\lambda$ la chaleur latente de transition de phase,
et $\eta$ la fraction solide en fonction de la temp\'erature $\Theta$:
\begin{equation*}
  \eta(\Theta) = \left\{
  \begin{array}{ll}
    1,&\quad \Theta < \Theta_t,\\
    0,&\quad \Theta \geq \Theta_t.
  \end{array}
  \right.
\end{equation*}
On notera dor\'enavant la chaleur sp\'ecifique en fonction de la
temp\'erature:
\begin{equation*}
  C_p(\Theta) = \left\{
  \begin{array}{ll}
    C_{p,1},&\quad \Theta < \Theta_t,\\
    C_{p,2},&\quad \Theta \geq \Theta_t.
  \end{array}
  \right.
\end{equation*}
De m\^eme, on note $\kappa_1$ et $\kappa_2$ les coefficients de
conduction thermique des phases solide et liquite, et on note $\kappa$
le coefficient de conduction en fonction de la temp\'erature:
\begin{equation*}
  \kappa(\Theta) = \left\{
  \begin{array}{ll}
    \kappa_1,&\quad \Theta < \Theta_t,\\
    \kappa_2,&\quad \Theta \geq \Theta_t.
  \end{array}
  \right.
\end{equation*}


On introduit l'enthalpie par unit\'e de volume du mat\'eriau $h$ en
fonction de la temp\'erature $\Theta$:
\begin{equation}
  H(\Theta) = \int_0^\Theta \rho C_p(s)\,\mathrm ds + \rho \lambda \parent{1 - \eta(\Theta)}.
\end{equation}

Soit $\beta:\mathbb R\to\mathbb R$ la fonction d\'efinie par
\begin{equation*}
  \beta(h) = \left\{
  \begin{array}{ll}
    \displaystyle\frac{h}{\rho C_{p,1}},& \quad \forall\ h < \Theta_t \rho C_{p,1},\\
    \Theta_f,& \quad \forall\ \Theta_t \rho C_{p,1} \leq h \leq \Theta_t \rho C_{p,1} + \rho\lambda,\\
    \Theta_f + \displaystyle\frac{h - \parent{\Theta_t \rho C_{p,1} + \rho\lambda}}{\rho
      C_{p,2}},& \quad \forall\ \Theta_t \rho C_{p,1} + \rho\lambda < h.
  \end{array}
  \right.
\end{equation*}

On note que la fonction $\beta$ est l'inverse de la fonction $h$ au
sens que:
\begin{equation*}
  \beta(H(\Theta)) = \Theta,\quad \forall \Theta.
\end{equation*}

On note $\partial \Omega = \Gamma_D \cup \Gamma_N$, tel que
$\Gamma_D\cap \Gamma_N = \emptyset$, o\`u $\Gamma_N$ et $\Gamma_D$
sont donn\'es. On suppose que le mat\'eriau occupe le domaine ouvert
$\Omega = (0, \infty)$. On note $u(t, x)$ l'enthalpie du mat\'eriau au
point $x\in\Omega$ et \`a l'instant $t > 0$. Le probl\`eme de Stefan
consiste \`a chercher une fonction $u:[0, \infty)\times\Omega$
solution de l'\'equation:
\begin{equation}\label{eq:stefan}
  \begin{array}{ll}
    \displaystyle\frac{\partial u}{\partial t} - \displaystyle\frac{\partial }{\partial x}\parent{\kappa \displaystyle\frac{\partial}{\partial x} \beta(u)} = 0,& x\in\Omega,\ t > 0,\\[\normalbaselineskip]
    u(0, x) = u_0(x),& x\in \Omega,\\[\normalbaselineskip]
    \beta(h) = g_D(t, x),& x \in\Gamma_D, t > 0\\[\normalbaselineskip]
    \displaystyle\frac{\partial \beta(h)}{\partial n}(t, x) = g_N(t, x),& x \in\Gamma_N, t > 0.
  \end{array}
\end{equation}
Le probl\`eme (\ref{eq:stefan}) \'etant singulier, il est \`a comprendre
au sens faible. Le probl\`eme faible correspondant \`a l'\'equation
(\ref{eq:stefan}) est le suivant. On cherche une fonction $u:[0,
  \infty)\times\Omega\to\mathbb R$ telle que:
\begin{equation}\label{eq:stefan-weak}
  \int_\Omega \frac{\partial u}{\partial t} v\,\mathrm dx +
  \int_\Omega \nabla \beta(u) \cdot \nabla v\,\mathrm dx = \int_{\Gamma_D} g_N v\, \mathrm d\sigma
\end{equation}
et
\begin{align*}
  u(0, x) &= u_0(x), \quad \forall x \in \Omega,\\
  u(t, x) &= g_D(t, x), \quad \forall x \in \Gamma_D,
\end{align*}
pour toute fonction $v:\Omega\to\mathbb R$ telle que
\begin{equation*}
  v(t, x) = 0, \quad \forall x \in \Gamma_D
\end{equation*}
suffisamment lisse.

Dans le cas o\`u on s'int\'eresse au probl\`eme 3D avec une
symm\'etrie sph\'erique, on utilise la formulation faible suivante:
\begin{equation}\label{eq:stefan-weak-spherical}
  \int_\Omega \frac{\partial u}{\partial t} vr^2\,\mathrm dr +
  \int_\Omega \partial_r \beta(u) \cdot \partial_r vr^2\,\mathrm dr =
  \int_{\Gamma_D} g_N vr^2 \, \mathrm d\sigma
\end{equation}
o\`u $u$ et $g_N$ sont fonction du rayon $r$ et du temps $t$, et $v$
est fonction de $r$.

\section{Discr\'etisation}\label{sec:discretisation}
On discr\'etise le probl\`eme (\ref{eq:stefan}) en temps et en espace
en suivant le travail de \cite{Paolini1988}. Soit $\tau > 0$ le pas de
temps, $t^k = k\tau$, $k = 0,1,2,\dots$ la discr\'etisation en temps
et $\mu$ une constante telle que:
\begin{equation}
0 < \mu \leq \frac{1}{\sup \beta'(h)}.
\end{equation}
Soit $k \in \mathbb N$, et on note $\Theta^k$ l'approximation de
$\Theta(t^k, .)$, et de m\^eme pour $u^k$
l'approximation de $u(t^k, .)$. On pose $\Theta^0(x) =
\Theta_0(x)$ et $u^0 = H(\Theta^0)$.  A chaque pas de temps,
l'\'equation (\ref{eq:stefan}) est lin\'earis\'ee de la mani\`ere
suivante.
\begin{align}
  \Theta^{k+1} - \frac{\tau}{\mu}\nabla\parent{\kappa\nabla\Theta^{k+1}} = \beta(u^k),\label{eq:lin-stefan}\\
  u^{k+1} = u^{k} + \mu\parent{\Theta^{k+1}-\beta(u^{k})}\label{eq:lin-chernoff}
\end{align}
avec les conditions de bord:
\begin{align}
  \Theta^{k+1}(x) = g_D(t^{k+1}, x),\quad& x \in\Gamma_D\\
  \displaystyle\frac{\partial \Theta^{k+1}}{\partial n}(t, x) = g_N(t^{k+1}, x),&\quad x \in\Gamma_N.
\end{align}

L'\'equation (\ref{eq:lin-stefan}) est discr\'etis\'ee par une
m\'ethode \'elements finis $\mathbb P_1$-$\mathbb P_1$ sur un maillage
uniforme d'un sous-ensemble born\'e $\Omega' \subset
\Omega$. Typiquement, $\Omega' = [0, L]$ avec $L > 0$.  L'\'equation
(\ref{eq:lin-chernoff}) est une simple \'equation alg\'ebrique, et la
somme est effectu\'ee degr\'e de libert\'e par degr\'e de libert\'e,
de m\^eme que l'\'evaluation de $\beta(u^k)$.


\section{Le package \texttt{freeze}}
Le code distribu\'e dans le package \texttt{freeze} impl\'emente le
sch\'ema num\'erique d\'ecrit \`a la section
\ref{sec:discretisation}. Cette section contient des informations
quant \`a l'installation du package, des d\'ependances requises, de
l'\'execution du mod\`ele et de l'acc\`es aux r\'esultats.

\subsection{Installation}
La compilation du code n\'ecessite les d\'ependance suivantes:
\begin{itemize}
  \item \texttt{petsc} version 3.7.6. Les versions ult\'erieures sont
    probablement support\'ees, mais sans garanties. Les version 3.6.3
    et ant\'erieures ne sont \textbf{pas} support\'ees.
  \item \texttt{liblapacke-dev} version 3.5.0. Cette version
    correspond probablement aux packages distribu\'es par les
    distributions de linux principales (Ubuntu, debian, ...)
  \item \texttt{openmpi}, par exemple, les packages n\'ecessaire dans les d\'epots
    Ubuntu sont \texttt{openmpi-bin}, \texttt{openmpi-common} et
    \texttt{libopenmpi-dev},
  \item \texttt{clang++} version 3.8.0. D'autres compilateurs sont
    probablement support\'es, pour autant qu'ils impl\'ementent le
    standard \texttt{C++14}. En revanche, en raison d'un bug dans
    \texttt{g++}, celui-ci n'est pas support\'e.
\end{itemize}

Pour compiler le package, il suffit de se placer dans le dossier
\texttt{freeze} et de lancer \texttt{make}:
\begin{lstlisting}[language={},frame=single,basicstyle=\ttfamily\footnotesize]
  [user@localhost ~/dev/freeze> make
\end{lstlisting}
Une fois la compilation termin\'ee, les ex\'ecutables se trouvent dans
le sous-dossier \texttt{freeze/bin/}. Il y en a deux:
\begin{itemize}
\item \texttt{freeze}: calcule la solution transitoire d'un probl\`eme
  de Stefan, selon les param\`etres sp\'ecifi\'es dans le fichier
  pass\'e en ligne de commande,
\item \texttt{neumann}: calcule la solution au temps final d'un probl\`eme
  de Stefan, selon les param\`etres sp\'ecifi\'es dans le fichier
  pass\'e en ligne de commande, puis calcul l'erreur $L_2$ entre le
  r\'esultat obtenu et la solution exacte de Neumann du probl\`eme
  adimentionn\'e.
\end{itemize}
On d\'ecrit l'usage de ces deux ex\'ecutables dans les parties qui suivent.

\subsection{Execution des mod\`eles: \texttt{freeze}}
L'ex\'ecutable \texttt{freeze} calcule la solution transitoire d'un
probl\`eme de Stefan. La sp\'ecification pr\'ecise se fait par
l'interm\'ediaire de fichier de configuration qui sont pass\'e en
param\`etres sur la ligne de commande.

Des exemples de fichier de configuration peuvent \^etre trouv\'es dans
le sous-dossier \texttt{freeze/bin}. Par exemple, la commande suivante:
\begin{lstlisting}[language={},frame=single,basicstyle=\ttfamily\footnotesize]
  [user@localhost ~/dev/freeze/bin> ./freeze cryolite-particle-remelt.conf
\end{lstlisting}
calcule la solution transitoire d'une situation qui correspond \`a la
formation initiale d'une couche de bain gel\'e \`a la surface d'une
particule, et qui finit par refondre.

L'execution de cette commande produit (potentiellement, en fonction
des param\`etres sp\'ecifi\'es dans
\texttt{cryolite-particle-remelt.conf}) en output une s\'erie de
fichiers \texttt{ASCII} dans le sous-dossier
\texttt{freeze/bin/output/}:
\begin{itemize}
\item \texttt{cryolite-particle-remelt-beta.dat}: le graph de la
  fonction $\beta$ autour de la transition de phase,
\item \texttt{cryolite-particle-remelt-neumann-exact-solution.dat}: le
  graph de la solution exact de Neumann \'evalu\'ee au temps final,
\item \texttt{cryolite-particle-remelt-ts-[0-9]+.dat}: la solution num\'erique
  transitoire \`a chaque pas de temps,
\item \texttt{cryolite-particle-remelt.dat}: la solution num\'erique
  au temps final,
\item \texttt{cryolite-particle-remelt-transitions.dat}: la position
  de la transition de phase en fonction du temps.
\end{itemize}
Les donn\'ees de ces fichiers peuvent \^etre visualis\'e par
\texttt{GNUPlot}, \texttt{MatLAB}, \texttt{GNU Octave} ou \'equivalent
sans post-processing n\'ecessaire.


L'output sur la sortie standard est minimal, et indique uniquement
l'instance de la collection de param\`etre sp\'ecifi\'ee dans le
fichier de param\`etre qui est en cours de calcul. Pour l'exemple
ci-dessus, on obtient:
\begin{lstlisting}[language={},frame=single,basicstyle=\ttfamily\footnotesize]
  [user@localhost ~/dev/freeze/bin> ./freeze cryolite-particle-remelt.conf
  running parameter set collection #1 of 1
  [user@localhost ~/dev/freeze/bin>
\end{lstlisting}
puisque la collection de param\`etre est un singleton dans ce cas-ci.

Un exemple de collection non triviale peut \^etre trouv\'e dans le
fichier de param\`etre
\texttt{cryolite-particle-remelt-study.conf}. Le lecteur int\'eress\'e
par la syntaxe et la s\'emantique de ces fichiers de param\`etres se
r\'ef\'erera \`a la documentation du package \texttt{parameter}.

\subsection{Execution des mod\`eles: \texttt{neumann}}
L'executable \texttt{neumann} effectue les m\^emes op\'erations que
l'executable \texttt{freeze}, mais calcule en plus l'erreur $L_2$
entre la solution num\'erique au temps final et la solution exacte de
Neumann. Les param\`etres sp\'ecifi\'es dans les fichier de
configuration doivent donc correspondre \`a la solution de Neumann
pour que l'erreur calcul\'ee ait un sens. Le lecteur int\'eress\'e une
discussion d\'etaill\'ee de cette solution dans \cite{Hill1987}.

Le fichier de param\`etre \texttt{stephan-exact-neumann-solution.conf}
correspond \`a cette configuration. L'output de l'ex\'ecutable
\texttt{neumann} est le m\^eme que celui de \texttt{freeze} en terme de
fichiers \texttt{ASCII}, mais affiche en plus l'erreur $L_2$ sur la
sortie standard \`a l'issue du calcul.

Le fichier \texttt{stephan-exact-neumann-solution.conf} sp\'ecifie une
collection de param\`etres qui correspondent \`a une \'etude de
convergence de l'erreur $L_2$. Par exemple, on obtient:
\begin{lstlisting}[language={},frame=single,basicstyle=\ttfamily\footnotesize]
  [user@localhost ~/dev/freeze/bin> ./neumann stephan-exact-neumann-solution.conf
  running parameter set collection #1 of 6
  L2_error(t = 0.5) = 0.00995969
  running parameter set collection #2 of 6
  L2_error(t = 0.5) = 0.00565716
  running parameter set collection #3 of 6
  L2_error(t = 0.5) = 0.00343643
  running parameter set collection #4 of 6
  L2_error(t = 0.5) = 0.00201683
  running parameter set collection #5 of 6
  L2_error(t = 0.5) = 0.00121125
  running parameter set collection #6 of 6
  L2_error(t = 0.5) = 0.000713004    
  [user@localhost ~/dev/freeze/bin>
\end{lstlisting}
